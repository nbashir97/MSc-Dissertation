\section{Summary of findings}

In this study, we analysed complex survey data representative of 36 million children in the US population to evaluate the association between immigration and dental caries. We found that being an immigrant was associated with a 140\% increase in the odds of having dental caries. In addition, race/ethnicity, household education, and FIPR were also significantly associated with risk of disease. Approximately speaking, the change in magnitude of risk associated with being an immigrant was comparable to that of a 300 percentage point change in FPIR.

\section{Interpretation of findings}

The findings of this study are in accordance with the existing literature, which suggests that immigration is a major determinant of health \citep{abubakr2018}. Given that the causal determinants for dental caries are sugar consumption and oral hygiene practices (including low fluoride exposure), the question arises: how does immigration have a bearing on these two factors?

Recent evidence shows that immigrants in the US place a high value on fresh, healthy foods, but they face a number of barriers to accessing these foods, such as income, mobility, and lack of time \citep{claussen2022}. Energy-dense foods (i.e., those high in added sugars) are typically cheaper than nutrient-dense foods, and sugar consumption is directly related to the amount of income which can be allocated to food expenditure \citep{drewnowski2004}. This means that high-sugar foods are inevitably a more accessible option to those groups of the population where cost is a barrier to adopting a healthy diet. Furthermore, individuals from migrant backgrounds tend to have lower levels of oral health-related knowledge and more negative oral health-related behaviours \citep{valdez2022}. This means they are less likely to understand why a low sugar consumption is so critical for preventing dental caries, even though they may perceive themselves to be making health-conscious choices in their diet. Many of these same points are also closely related to oral hygiene practices. Due to this reduced level of oral health-related knowledge, migrants may be less likely to appreciate the role of good oral hygiene practices and fluoride exposure for disease prevention. Again, this will increase the risk of disease, even though these individuals might feel that they are taking all the necessary precautions to maintain a high standard of oral health. 

Another key consideration is the fact that children who are migrants often have barriers to accessing a high-quality education, hence these individuals are less likely to be informed on the benefits of a healthy lifestyle, and how to go about maintaining a high standard of health \citep{abubakr2018}. One point of interest here is the so-called "immigrant paradox", whereby it has been observed that children from migrant backgrounds tend to outperform their peers in school \citep{crosnoe2011}. Therefore, the challenge is not necessarily in motivating migrant children who are currently in education, but in ensuring that all migrant children have the same opportunities for accessing education in the first place. As well as reduced access to education, immigrants also have reduced access to healthcare services, due to a wide array of social, political, and economic factors \citep{khullar2019}. This has two implications: (i) migrants may be less likely to receive the necessary information from dental professionals informing them how to prevent the development of oral disease, (ii) migrants may be less likely to receive timely diagnosis and treatment for dental caries.

\section{Strengths and limitations}

The strengths of this study lie in the robust sampling design and methodology of NHANES. Within NHANES, a rich set of clinical and demographic data pertaining to the sampled individuals are collected, which allows for assessment of the association between immigration and dental caries in the absence of many key confounders. Furthermore, due to the complex survey design, the findings of these analyses can be generalised to the wider US paediatric population. Finally, the use of a full-mouth examination protocol is highly valuable as it ensures the clinical diagnoses made are accurate. It is quite uncommon for a sample of this size to have full-mouth records available, due to the great deal of time and expense involved in collecting such data.

There are also limitations to this study which should be noted. First, NHANES is a cross-sectional study which means that it essentially records a snapshot of the population at one period of time. Since there is no longitudinal data, this makes inferring causality highly dubious; at best, we can only say that there is a strong association between immigration and dental caries, rather than a true causal relationship. Secondly, this study is assessing the presence of active dental caries, but looking at historic evidence of disease experience (e.g., the presence of fillings or missing teeth) may alter this relationship. Of course, it must be remembered that teeth may be filled or missing for reasons other than dental caries. Finally, a complete-case analysis was carried out here (i.e., individuals with missing data were excluded). If there are systematic differences between excluded and included individuals then this can bias the findings, and alternative approaches such a multiple imputation are possible. It should be noted that the appropriate use of sampling weights should largely mitigate this as they are specifically designed to adjust for the nonresponse factor, whereas imputation methods require assumptions to be made about the nature of the missing data (see \autoref{eq:final-wt}).

\section{Implications of findings}

Future research may wish to probe the factors mediating the relationship between immigration and dental caries. We postulate that it is likely to be inadequate dietary and oral hygiene practices, further compounded by lack of access to adequate health-related education, which explain the observed relationship. By understanding the factors which bring this about, it is then possible for more targeted efforts to be made in order to reduce the risk of disease amongst migrant children. This work also has implications for policy makers, who may wish to allocate healthcare resources to tackle the disparity in risk of disease. This could be done by improving the educational resources available to those from migrant backgrounds, in order to mitigate the lower levels of oral health-related knowledge, as well as improving accessibility to care in order to achieve more equitable health outcomes. However, before implementation of pro-immigrant policies such as these, it would first be worth considering reversing some of the current anti-immigrant policies in the US which are known to have direct and substantial impacts on migrant health \citep{nuño2022}. In fact, one of the biggest challenges for policy makers today is that, with anti-immigrant agendas becoming more mainstream, policies which are seen to allocate taxpayers' money to non-natives may be perceived as highly inflammatory by some members of the public.