The work presented in this dissertation is both novel and highly informative, in that it quantifies the association between immigration and dental caries amongst children in the US. We have explored many of the key considerations that must be accounted for when analysing data from complex surveys and, in doing so, we ensure that our findings can be generalised beyond the analytical sample and to the wider US population. At the same time, we see that statistical theory for survey data analysis, particularly the methods for model-based analyses, still possess fundamental limitations. These range from a lack of robust options for testing model fit through to a dependence on many (often untestable) assumptions, particularly when compared to design-based analyses. Given the extent to which CSS data is used to answer questions across a wide range of scientific disciplines, a call for new methodological developments is certainly warranted.

In summary, our results indicate that immigration appears to be associated with a profound increase in the risk of dental caries amongst children - a 140\% increase to be exact. The most contemporary statistics on the epidemiology of dental caries in the US show a profound social disparity in how the disease is distributed; the findings of this work only further strengthen the pressing need for public health reform in order to address the socioeconomic determinants of oral health \citep{bashir2021, bashir2022}. Therefore, this work has implications for motivating further research and also in guiding policy makers with regard to allocation of healthcare resources. Future researchers may wish to decompose the exact nature of the relationship between immigration and oral health, in order to precisely define the factors which mediate the association. Public health strategy can then be made more efficient by adopting a common risk factor approach, where the aim would be to address those features in the causal pathway that simultaneously increase immigrants' risk of both oral and systemic diseases \citep{sheiham2000}. 

However, even with this roadmap in mind, the salient issue for policy makers will be navigating the complex political landscape with regards to the rights of immigrants in modern society.